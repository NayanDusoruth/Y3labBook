% Options for packages loaded elsewhere
% Options for packages loaded elsewhere
\PassOptionsToPackage{unicode}{hyperref}
\PassOptionsToPackage{hyphens}{url}
\PassOptionsToPackage{dvipsnames,svgnames,x11names}{xcolor}
%
\documentclass[
  letterpaper,
  DIV=11,
  numbers=noendperiod]{scrartcl}
\usepackage{xcolor}
\usepackage{amsmath,amssymb}
\setcounter{secnumdepth}{-\maxdimen} % remove section numbering
\usepackage{iftex}
\ifPDFTeX
  \usepackage[T1]{fontenc}
  \usepackage[utf8]{inputenc}
  \usepackage{textcomp} % provide euro and other symbols
\else % if luatex or xetex
  \usepackage{unicode-math} % this also loads fontspec
  \defaultfontfeatures{Scale=MatchLowercase}
  \defaultfontfeatures[\rmfamily]{Ligatures=TeX,Scale=1}
\fi
\usepackage{lmodern}
\ifPDFTeX\else
  % xetex/luatex font selection
\fi
% Use upquote if available, for straight quotes in verbatim environments
\IfFileExists{upquote.sty}{\usepackage{upquote}}{}
\IfFileExists{microtype.sty}{% use microtype if available
  \usepackage[]{microtype}
  \UseMicrotypeSet[protrusion]{basicmath} % disable protrusion for tt fonts
}{}
\makeatletter
\@ifundefined{KOMAClassName}{% if non-KOMA class
  \IfFileExists{parskip.sty}{%
    \usepackage{parskip}
  }{% else
    \setlength{\parindent}{0pt}
    \setlength{\parskip}{6pt plus 2pt minus 1pt}}
}{% if KOMA class
  \KOMAoptions{parskip=half}}
\makeatother
% Make \paragraph and \subparagraph free-standing
\makeatletter
\ifx\paragraph\undefined\else
  \let\oldparagraph\paragraph
  \renewcommand{\paragraph}{
    \@ifstar
      \xxxParagraphStar
      \xxxParagraphNoStar
  }
  \newcommand{\xxxParagraphStar}[1]{\oldparagraph*{#1}\mbox{}}
  \newcommand{\xxxParagraphNoStar}[1]{\oldparagraph{#1}\mbox{}}
\fi
\ifx\subparagraph\undefined\else
  \let\oldsubparagraph\subparagraph
  \renewcommand{\subparagraph}{
    \@ifstar
      \xxxSubParagraphStar
      \xxxSubParagraphNoStar
  }
  \newcommand{\xxxSubParagraphStar}[1]{\oldsubparagraph*{#1}\mbox{}}
  \newcommand{\xxxSubParagraphNoStar}[1]{\oldsubparagraph{#1}\mbox{}}
\fi
\makeatother


\usepackage{longtable,booktabs,array}
\usepackage{calc} % for calculating minipage widths
% Correct order of tables after \paragraph or \subparagraph
\usepackage{etoolbox}
\makeatletter
\patchcmd\longtable{\par}{\if@noskipsec\mbox{}\fi\par}{}{}
\makeatother
% Allow footnotes in longtable head/foot
\IfFileExists{footnotehyper.sty}{\usepackage{footnotehyper}}{\usepackage{footnote}}
\makesavenoteenv{longtable}
\usepackage{graphicx}
\makeatletter
\newsavebox\pandoc@box
\newcommand*\pandocbounded[1]{% scales image to fit in text height/width
  \sbox\pandoc@box{#1}%
  \Gscale@div\@tempa{\textheight}{\dimexpr\ht\pandoc@box+\dp\pandoc@box\relax}%
  \Gscale@div\@tempb{\linewidth}{\wd\pandoc@box}%
  \ifdim\@tempb\p@<\@tempa\p@\let\@tempa\@tempb\fi% select the smaller of both
  \ifdim\@tempa\p@<\p@\scalebox{\@tempa}{\usebox\pandoc@box}%
  \else\usebox{\pandoc@box}%
  \fi%
}
% Set default figure placement to htbp
\def\fps@figure{htbp}
\makeatother





\setlength{\emergencystretch}{3em} % prevent overfull lines

\providecommand{\tightlist}{%
  \setlength{\itemsep}{0pt}\setlength{\parskip}{0pt}}



 


\KOMAoption{captions}{tableheading}
\makeatletter
\@ifpackageloaded{caption}{}{\usepackage{caption}}
\AtBeginDocument{%
\ifdefined\contentsname
  \renewcommand*\contentsname{Table of contents}
\else
  \newcommand\contentsname{Table of contents}
\fi
\ifdefined\listfigurename
  \renewcommand*\listfigurename{List of Figures}
\else
  \newcommand\listfigurename{List of Figures}
\fi
\ifdefined\listtablename
  \renewcommand*\listtablename{List of Tables}
\else
  \newcommand\listtablename{List of Tables}
\fi
\ifdefined\figurename
  \renewcommand*\figurename{Figure}
\else
  \newcommand\figurename{Figure}
\fi
\ifdefined\tablename
  \renewcommand*\tablename{Table}
\else
  \newcommand\tablename{Table}
\fi
}
\@ifpackageloaded{float}{}{\usepackage{float}}
\floatstyle{ruled}
\@ifundefined{c@chapter}{\newfloat{codelisting}{h}{lop}}{\newfloat{codelisting}{h}{lop}[chapter]}
\floatname{codelisting}{Listing}
\newcommand*\listoflistings{\listof{codelisting}{List of Listings}}
\makeatother
\makeatletter
\makeatother
\makeatletter
\@ifpackageloaded{caption}{}{\usepackage{caption}}
\@ifpackageloaded{subcaption}{}{\usepackage{subcaption}}
\makeatother
\usepackage{bookmark}
\IfFileExists{xurl.sty}{\usepackage{xurl}}{} % add URL line breaks if available
\urlstyle{same}
\hypersetup{
  colorlinks=true,
  linkcolor={blue},
  filecolor={Maroon},
  citecolor={Blue},
  urlcolor={Blue},
  pdfcreator={LaTeX via pandoc}}


\author{}
\date{}
\begin{document}


\section{2.1 Lab session notes}\label{lab-session-notes}

\subsection{28/01/2026 note:}\label{note}

Have begun switching from onenote labbook to a quarto document. Still
hand copy equations for ``Wk0.5 background theory and concept''.

\subsection{29/01/2026 lab log}\label{lab-log}

Turci session; went over Jaynes model mean field theory derivion

catchup w/lab partner; outlined code She summarised lit review:

\begin{itemize}
\tightlist
\item
  cluster definition:

  \begin{itemize}
  \tightlist
  \item
    useful definition; connected points w/same spin; (domb and stall,
    link 2)
  \end{itemize}
\end{itemize}

code plan:

\begin{itemize}
\tightlist
\item
  dataset object

  \begin{itemize}
  \tightlist
  \item
    list of config objects over temperature
  \item
    save as folder of config objects - give config objects useful names
    to reconstruct dataset object
  \end{itemize}
\item
  getClusters - returns list of clusters as some sort of data structure
  (maybe True/False bool array w/spin value?)
\item
  cluster analysis methods:

  \begin{itemize}
  \tightlist
  \item
    cluster size
  \item
    correlation length?
  \item
    cluster mean position? may not be useful for debugging?
  \item
    cluster spread (standard deviation) - measure of how spread out
  \end{itemize}
\end{itemize}

cluster selection algorithm outline:

may need two methods; one ``find cluster from point'' method and one
``find all clusters'' method

findAllClusters():

\begin{itemize}
\tightlist
\item
  create bool array of ``checked'' values (True represents cells that
  haven't been checked yet, False represents cells that have been
  checked)
\item
  while loop:

  \begin{itemize}
  \tightlist
  \item
    pick random ``True'' cell; assign to array indices
  \item
    propagate cluster from cell
  \item
    set ``checked values'' array position corresponding to new cluster
    to false
  \end{itemize}
\end{itemize}

findCluster(cell, config); do as recursive algorithm

checked algorithm idea w/Scarlett; paper
(https://journals.aps.org/prl/pdf/10.1103/PhysRevLett.62.361)

reorganised code into file structure - Main - cluster - testing

things that need doing: - cluster algorithm implementations - dataset
object implementation - code cleanup

Scarlett will do cluster stuff, i'll do dataset and code cleanup stuff.

got basic dataset constructor and saving/read to file functionality
working

dataset has a set of:

\begin{itemize}
\tightlist
\item
  equilibration steps - number of steps taken to equilibrate simulation
  - doesn't save configs
\item
  simulation steps - number of steps during actual simulation
\end{itemize}

observables will be average of values over last simulation steps.
implemented partition function, entropy, helmholtz free energy, as
observables. implemented ``average'' observable methods - they get the
average of a given observable between simulation indices Dataset can get
standard observables from config now

\subsection{stuff for next week:}\label{stuff-for-next-week}

\begin{itemize}
\tightlist
\item
  Scarlett continues clustering analysis
\item
  I continue dataset stuff and main code cleanup

  \begin{itemize}
  \tightlist
  \item
    NEED to comment/document out code
  \end{itemize}
\item
  i read up on theory
\end{itemize}

\subsection{01/02/2026 notes:}\label{notes}

Read up on theory Scarlett reviewed during Wk1 and added relevant theory
to (Wk 2.2 Additional reading). Have further read into some of the
critical exponents theory as well in order to develop an understanding
of how to compute them from our model. The critical exponent theory
follows heavily from cite\{CritExpLink3\} which seems to handle scaling
phenomena. As it stands, there may be two approaches to computing the
critical exponents:

\begin{itemize}
\tightlist
\item
  Simple case; vary models over temperature and fit some variation of
  \((T-T_{c})^{x}\) where x is a critical parameter to various
  observables (as seen in cite\{ComplexityAndCriticality\} P132-134)
\item
  Attempt to follow the procedure done by cite\{CritExpLink3\}
\end{itemize}

The former option is simpler and is a natural extension of current work
w/dataset object; however am unsure if it'd accomodate finite scaling
effects. It may be worth implementing the former case first in order to
test the idea; furthermore graphing observables over temperature may
give a better idea of the dynamics being explored; so even if this
simpler method fails, it may yield experience and insight into the
system's behaviour.

Note reading around topic briefly the ``Swendsen-Wang'' and ``Wolff''
algorithms are mentioned; they seem to be used to improve compute times
near critical temperatures. Am unsure if they'd be necessary; but noting
their existence/potential use case here in case they'd be useful later.

A pure computational approach to extracting critical exponents may thus
follow:

\begin{itemize}
\tightlist
\item
  create datasets of dimension d, size L over temperature ranges - try
  and implement a ``critical temp finding'' method which performs a
  binary search for the critical temperature in a data range for a given
  observable
\item
  fit the simple critical exponent function to the data
\item
  perform above for a range of sizes for same dimension; compare results
\item
  if scaling seems predictable; computational approach may be
  sufficient, and exploration of higher dimension may begin
\item
  else will need to read up more on how to avoid scaling effects
\end{itemize}

Therefore to do list follows:

\begin{itemize}
\item
  dataset object: finish off; it should be able to return a pandas
  dataframe of observables over temperature
\item
  may be useful to generalise dataset object so it may vary model size
  instead of temperature
\item
  introduce plotting methods for observables
\item
  introduce ``critical temperature finding'' methods - comparing
  derivatives of observables at extremal ends of data, should be able to
  determine if current location is above/below critical temperature -
  this would allow a binary search for the critical temperature to
  arbitrary precision
\item
  fitting methods of observables to critical temp expressions
\item
  further cleanup of code
\end{itemize}

\subsection{02/02/2026:}\label{section}

tidied up dataset object - introduced a to dataframe method; and cleaned
up file saving system.

\subsubsection{dataset object testing}\label{dataset-object-testing}

Starting to generate experimental datasets to gain experience generating
large datasets (naming format; \emph{}:

\begin{itemize}
\tightlist
\item
  testData20x20\_1000\_100: small 20x20 dataset run over 1000 eq steps,
  and simulated for 100 steps; range of 20 beta \(\in [0, 3]\) -
  magnetisation doesn't plot neatly
\item
  testData50x50\_1000\_100: small 50x50 dataset run over 1000 eq steps,
  and simulated for 100 steps; range of 20 beta \(\in [0, 3]\) -
  magnetisation doesn't plot neatly
\end{itemize}

neither test case thus far plots neatly as seen:

\begin{figure}[H]

{\centering \pandocbounded{\includegraphics[keepaspectratio]{figures/earlyMagPlot.png}}

}

\caption{Fig 2.1 - Magnetisation plotted over \(\beta\) for
testData50x50\_1000\_100}

\end{figure}%

\begin{figure}[H]

{\centering \pandocbounded{\includegraphics[keepaspectratio]{figures/earlyAbsMagPlot.png}}

}

\caption{Fig 2.2 - Absolute Magnetisation plotted over \(\beta\) for
testData50x50\_1000\_100}

\end{figure}%

Which shows a general increase in absolute magnetisation over \(\beta\)
(contrary to what intuition would suggest, with magnetisation increasing
for low beta); with the occaisonal datapoint dropped to 0. Furthermore
it is odd how absolute magnetisation increases, then suddenly drops as
in fig 2.2

The final configuration states for this data was generated for visual
inspection; some examples include:

\pandocbounded{\includegraphics[keepaspectratio]{figures/magTestConfig0.pdf}}\hfill
\pandocbounded{\includegraphics[keepaspectratio]{figures/magTestConfig15.pdf}}\hfill
\pandocbounded{\includegraphics[keepaspectratio]{figures/magTestConfig16.pdf}}\hfill
\pandocbounded{\includegraphics[keepaspectratio]{figures/magTestConfig19.pdf}}\hfill

On initial inspection the low and high \(\beta\) configs appear
reasonable; fig.2.3 shows an orderly config for low \(\beta\) whilst fig
2.6 shows a relatively disorded config. However the intermediate
examples show a confusing switch in behaviour from relatively
disorganised to organised (these correspond to one of the drops in fig
2.2).

A few things become apparent comparing figs 2.3-2.6 to fig 2.1:

\begin{itemize}
\tightlist
\item
  magnetisation computation is going wrong somehwhere; as highly ordered
  states (I.E fig 2.3) are reading low magnetisation whilst disorded
  stats (I.E fig 2.5) are reading high magnetisation. This contradicts
  an understanding of the underlying theory that ordered ising models
  should be more magnetised
\item
  the source of the discontinuous drops are unclear - there is a major
  drop at \(\beta\approx 2.4\) alongside several smaller drops; if those
  smaller drops didn't exist, that large drop could be attributed to a
  phase transition
\end{itemize}

The random drops may be due to noise effects (the magnetistion is
averaged over the relatively few 100 simulation steps); however the fact
that the magnetisation grows over beta isn't clear.

Performing larger test:

\begin{itemize}
\tightlist
\item
  testData50x50\_1000\_1000: note over range \(\beta \in [0, 4]\)
\end{itemize}

am hoping this will fix some of the data variability and give a clearer
picture of phase transitions (even if the magnetisation over \(\beta\)
relation is flipped)

am an absolute idiot; \(\beta \propto \frac{1}{T}\); so lower \(\beta\)
cooresponds to higher \(T\); so the pattern seen is as expected -
nothing is broken\ldots{}

\subsubsection{additional code development
ideas}\label{additional-code-development-ideas}

May need to modify dataset object to allow different configs to have
different equilibration times; cite\{CritExpLink3\} suggests that
equilibration time may vary over parameters

Furthermore may need to implement system to estimate equilibration
times:

\begin{itemize}
\tightlist
\item
  dual system introduced by cite\{CritExpLink3\} - create 2 configs; one
  starting all spin up, one all spin down; compare magnetisations of two
  results until they are within error of each other
\item
  may be worth implementing as purpose built system - i suspect that the
  config code can be streamlined purely for the purpose of computing
  these equilibration times
\item
  may also be worth creating a ``equilibrationTime dataset'' - able to
  generate a set of dualConfigs identical to dataset parameters; and
  spit out equilibration times - maybe make it suck that it can linearly
  interpolate between equilibrationTime datapoints; would allow for
  sparser data
\end{itemize}




\end{document}
